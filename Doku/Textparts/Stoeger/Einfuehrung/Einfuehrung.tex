%\todo{add ausgangslage}
%\todo{add zielsetzung}
%\todo{add inspiriert von einem youtube video?}
\section{Ausgangslage}
Inspiriert von einem YouTube-Video\textsuperscript{\cite{YoutubeVideo}} entstand die Idee, ein Schlauchboot mit zwei Elektromotoren zu einem Luftkissenboot umzubauen. Dabei sollten die im Video angesprochenen Problem berücksichtigt und ausgebessert werden.

\section{Zielsetzung}
Das Ziel der Diplomarbeit ist es, ein Schlauchboot zu einem Luftkissenboot umzubauen. Das Hovercraft soll mit zwei 10kW Elektromotoren mit dazugehörigen Propeller konstruiert werden.\\
Das fertige Luftkissenboot soll in der Lage sein eine erwachsene Person über Land, sowie über Wasser transportieren zu können.
\todo{das klingt ned so toll}


\section{Individuelle Themenstellung}
\begin{itemize}
    \item \textbf{Stefan Deimel:}
    \begin{itemize}
      \item Konstruktion
      \item Motorregelung
    \end{itemize}
    \item \textbf{Philipp Eilmsteiner:}
    \begin{itemize}
      \item Steuerelektronik
      \item HMI
    \end{itemize}
    \item \textbf{Julia Stöger:}
    \begin{itemize}
      \item Sicherheitskonzept
      \item Ladeelektronik
    \end{itemize}
\end{itemize}



% Aus einem Schlauchboot wurde mit Hilfe von zwei Elektromotoren ein voll funktiontüchtiges Luftkissenboot gebaut. \\
% Dieses ist in der Lage, eine erwachsene Person mehrere Minuten, über dem Boden schwebend, zu befördern. \\
% Das Lustkissenboot lässt sich sowohl auf Land, als auch auf Wasser fahren. \\
