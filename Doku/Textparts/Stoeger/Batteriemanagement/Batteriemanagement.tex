\section{Ladeelektronik}
\subsection{Einleitung}
Nach längerer Überlegung wurde entschieden kein Batteriemanagement selbst zu entwerfen, sondern ein Ladegerät und ein Schaltnetzteil zuzukaufen. \\
Als Akkus wurden 2x vier, TopFuel LiPo 35C Power-X 7500mAh 7s in Serie geschaltet, siehe \ref{sec:Akkus}. \\

\textbf{Eine Anleitung für den Ladevorgang findet man im \autoref{sec: bilderanleitung_laden}.}

\missingfigure{Akkus}
\subsection{Ladegerät}
Verwendet wird das Ladegerät $"$iSDT SMART Dup Ladegerät P30 - 1500W, 30A$"$. Das Ladegerät ist ein 2-Kanalladegerät, das heißt, 
dass beide Akku-Stränge gleichzeitig geladen werden können. 
Das P30 kann mit dem Programm $"$Dual Task$"$ beide Kanäle mit dem selben Programm laufen lassen. Somit gibt man die Ladeparameter nur einmal ein. \\
Der maximale Ladestrom beträgt 30$\mathrm{A}$ pro Strang. \\
Weitere Infos sind unter \href{https://www.modell-hubschrauber.at/Ladegeraete-Netzteile-Ladekabel-und-Zubehoer/Ladegeraete/Ladegeraete-12Volt/iSDT-SMART-Dup-Ladegeraet-P30-1500W-30A-8S-Lipo::43075.html}{Link zur Website} 
zu finden. 
\missingfigure{Ladegeraet}

\subsection{Schaltnetzteil}
Nach gründlicher Überlegung fiel die Entscheidung auf das $"$iSDT SP3060 SMART POWER Schaltnetzteil$"$.
Das Schaltnetzteil ist gegen Überlast, Überspannung, Überhitzung und Kurzschluss geschützt. 

\begin{minipage}{13cm}
    \centering
    \includegraphics[width=0.5\textwidth]{Fotos/SP3060.png}
    \captionof{figure}{Schaltnetzteil}   
\end{minipage}

\newpage

