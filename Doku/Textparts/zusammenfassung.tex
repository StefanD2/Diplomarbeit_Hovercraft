%% start of file zusammenfassung.tex

\selectlanguage{naustrian}
\begin{abstract}
\responsible{Julia Stöger}
Das Ziel unserer Diplomarbeit ist es, aus einem Schlauchboot ein funktionstüchtiges elektrisches Luftkissenboot zu bauen, welches
sich sowohl am Land, als auch auf Wasser fortbewegen kann.\\
Mit einem Elektromotor und einen Propeller, welcher nach unten zeigt, wird ein Luftstrom nach unten erzeugt, auf dem das Boot schwebt. Der zweite Propeller zeigt nach hinten und ist somit für den Vortrieb verantwortlich. Um das Boot lenkbar zu machen, sind hinter dem hinteren Propeller servo-gesteuerte Fahnen zum Umlenken des Luftstroms.\\
Der Fahrer ist in der Lage, das Luftkissenboot mit einem Lenker und zwei Daumengashebeln zu steuern.

% Das Hovercraft hat 2 Motoren und zugehörige Propeller, einen für den Auftrieb, um den Schwebezustand zu erreichen, und einen für den Antrieb.  \\
% Die Ansteuerung der beiden Motoren erfolgt über je ein Daumengas, welche an dem Lenker befestigt sind. \\
% Der Aufbau auf dem Boot wurde aus Holz gebaut. Elektrische Komponenten, wie Motoren, Propeller und ähnliches wurden zugekauft. \\
% Für das Aufladen der Akkumulatoren wird ein Ladegerät angeschlossen, sodass die Akkus nicht aus dem Hovercraft 
% entfernt werden müssen. 
\end{abstract}

%% end of file zusammenfassung.tex