\section{Arduino Bibliotheken}
Im Zuge der Programmierung der Arduinos wurde auf eine Vielzahl von bereits vorhandenen Bibliotheken zurückgegriffen. Konkret wurden für die CAN-Bus Module\textsuperscript{\cite{ArduinoBibliothekCANBus}}, 
das Servo-Driver-Modul\textsuperscript{\cite{ArduinoBibliothekServoPWM}}, 
die ADS1115 Module\textsuperscript{\cite{ArduinoBibliothekADS1115}},
das GPS-Modul\textsuperscript{\cite{ArduinoBibliothekGPS}},
sowie für die Kommunikation mit dem Nextion Display\textsuperscript{\cite{ArduinoBibliothekNextion}}, frei im Internet zugängliche Bibliotheken verwendet.
\subsection{Jeti Decoder}
Hierbei handelt es sich um eine selbstgeschriebene Arduino-Bibliothek, mit welcher die über eine serielle Verbindung zur Verfügung gestellten Telemetrie-Daten der Motorregler ausgelesen werden können.
\subsubsection{Jeti.hpp}
\lstinputlisting{../Programmierung/libs/Jeti/Jeti_UNO.hpp}
\subsubsection{Jeti\_ ATmega\_Serial1.hpp}
\lstinputlisting{../Programmierung/libs/Jeti/Jeti_ATmega_Serial1.hpp}
\subsubsection{Jeti\_UNO.hpp}
\lstinputlisting{../Programmierung/libs/Jeti/Jeti_UNO.hpp}
\subsubsection{JetiBase.hpp}
\lstinputlisting{../Programmierung/libs/Jeti/JetiBase.hpp}
\subsubsection{JetiBase.cpp}
\lstinputlisting{../Programmierung/libs/Jeti/JetiBase.cpp}

\subsubsection{JetiModes.hpp}
\lstinputlisting{../Programmierung/libs/Jeti/JetiModes.hpp}