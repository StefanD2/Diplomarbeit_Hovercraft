
\section{Sicherheitsmaßnahmen}
Um die Sicherheit während der Benützung des Luftkissenboots zu gewährleisten wurden mehrere Sicherheitsvorkehrungen getroffen.
\subsection{Not-Aus-Schalter mit Schlüssel}
Die Notabschaltung der Regler, und damit auch der Motoren, erfolgt durch Betätigen des Not-Aus-Schalters, da dies den Haltestrom des Hauptrelais unterbricht. 
Es ist möglich das Hovercraft nach einer Notabschaltung mit Hilfe des Schlüssels wieder zu starten. 

\begin{minipage}{8cm}
    \centering
    \includegraphics[width=0.3\textwidth]{Fotos/Notaus.png}
    \captionof{figure}{Notausschlüsselschalter}    
\end{minipage}
  

\subsection{Daumengas für beide Propeller}
Durch Betätigen der beiden Daumengase, rechts und links am Lenker, erfolgt der Antrieb der Propeller. Das linke Daumengas ist für den Auftrieb, für den unteren Propeller und
das rechte für den Antrieb, den hinteren Propeller. Durch Loslassen des Daumengases bremst das Hovercraft sofort.

\subsection{Gitterabdeckungen}
Über den beiden Propellern befinden sich Gitter, die es unmöglich machen, bewusst oder unbewusst in ein drehendes Teil zu fassen. 
Weiters dienen die Gitter als Schutz vor losen Teilen, die sich in den Propellern verfangen könnten. 

\section{Gerätesicherheit}

\subsection{Bremsschutz}
Zum Schutz beim Bremsen wurde ein zusätzlicher Boden angebracht. Dieser wurde mit einem Kraftkleber und Lochband am Boot befestigt. Der zusätzliche Boden bietet 
Schutz vor 