\subsubsection*{1. Juni 2020 - Überprüfung des Konzeptes und der notwendigen Motorleistung mithilfe eines Prototypen}
Es wurde ein Prototyp gebaut, um die Motorleistung zu überprüfen und einen Schwebezustand zu erreichen.

\subsubsection*{1. Oktober 2020 - Propeller für Fortbewegung mit dazugehörigen Fahnen für Lenkung funktionstüchtig}
Die Lenkfunktion wurde mithilfe von drei Potentiometern realisiert und getestet. Mithilfe der Fahnen lässt sich das Luftkissenboot lenken. Um den Propeller für 
die Fortbewegung befestigen zu können, wurde ein Aufbau aus Holz gebaut.

\subsubsection*{1. Dezember 2020 - Boardelektronik und  Akkumanagement vollständig entwickelt}
Das Konzept und Design der gesamten Boardelektronik und des Akkumanagements wurde  vollständig entwickelt
\subsubsection*{1. Februar 2021 - Akkumanagement funktionstüchtig, getestet und im Luftkissenboot verbaut}
Anfangs war geplant ein Akkumanagement selbst zu bauen und planen, da wir uns jedoch umentschieden und ein Ladegerät zugekauft haben, konnte dieser Meilenstein um ein paar Tage 
nicht eingehalten werden. 
\subsubsection*{20. Februar 2021 - Gesamte Boardelektronik fertiggestellt}
Die komplette Boardelektronik ist funktionstüchtig, getestet und fertig im Luftkissenboot verbaut. 

\subsubsection*{1. März 2021 - Praktischer Teil der Diplomarbeit beendet}

\subsubsection*{15. März 2021 - Diplomarbeit fertig geschrieben}
