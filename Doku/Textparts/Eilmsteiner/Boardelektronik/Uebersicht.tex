\tikzmath{
    \test=0;
    \platinenX1=2.9;
    \platinenX2=9;
    \motorreglerX1=10.5;
    \motorreglerX2=15;
}

\resizebox{0.8\textwidth}{!}{
\begin{circuitikz}[loops/.style={circuitikz/inductors/coils=#1}]]

    % Analoge Eingänge der Lenkerplatine
    \newarray\analoginputs
    \readarray{analoginputs}{Daumengas 1&Daumengas 2&Lenker 1&Lenker 2}
    \foreach \r in {0,1,2,3}{
        \tikzmath{int \t; \t = \r+1;}
        \draw[-latex,thick,double] (6+\r*0.3,14.2) -- (6+\r*0.3,15.7-0.3*\r) node [at start,rotate=90,left] {\scriptsize A\r} -- (8,15.7-0.3*\r)  node[right] {\scriptsize \analoginputs(\t)};
    }


    % Platinenrahmen + Anschluss an CAN-Bus
    \foreach \i in {0,4,8,12}
        {\draw[thick] (0,\i) to [short,-] ++ (\platinenX1,0) node[right] {\scriptsize CAN+};
        \draw[thick] (0.3,\i+0.4) to [short,-] ++ (\platinenX1-0.3,0) node[right] {\scriptsize CAN-};
        \draw[thick] (0.6,\i+0.8) to [short,-] ++ (\platinenX1-0.6,0) node[right] {\scriptsize GND};
        \draw[thick] (0.9,\i+1.2) to [short,-] ++ (\platinenX1-0.9,0) node[right] {\scriptsize VCC};
        \draw[very thick] (\platinenX1,\i+2.2) rectangle (\platinenX2,\i-1);}

    % CAN-Bus Leitungen
    \draw[thick] (0.0,0) to [short,-*] ++ (0,4) to [short,-*] ++ (0,4) to [short,-] ++ (0,4);
    \draw[thick] (0.3,0.4) to [short,-*] ++ (0,4) to [short,-*] ++ (0,4) to [short,-] ++ (0,4);
    \draw[thick] (0.6,0.8) to [short,-*] ++ (0,4) to [short,-*] ++ (0,4) to [short,-*] ++ (0,4);
    \draw[thick] (0.9,1.2) to [short,-*] ++ (0,4) to [short,-*] ++ (0,4) to [short,-*] ++ (0,4);

    \draw[-latex,thick] (0.6,12+0.8) -- (0.6,15.5) -- (1.4,15.5) node[right] {\scriptsize CAN-GND};
    \draw[-latex,thick] (0.9,12+1.2) -- (0.9,15)  -- (1.4,15) node[right] {\scriptsize CAN-VCC};

    % Motorreglerplatine
    \foreach \i in {1,2}
    {\node [thick, fit={(2.9,4*\i+2.2) (\platinenX2,4*\i-1)},inner sep=0, align=center,shift={(0.2,0.9)}] {Platine Motoransteuerung\\ siehe \ref{sec:Motorregleransteuerung}};

    \draw [very thick] (\motorreglerX1,4*\i+1.4) rectangle (\motorreglerX2,4*\i-1);
    \node [thick, fit={(\motorreglerX1,4*\i+1.4) (\motorreglerX2,4*\i-1)},inner sep=0, align=center,shift={(0.4,0.0)}] {Master Spin\\ 220 Pro OPTO}; % Beschriftung Master Spin 220 Pro OPTO 

    \draw[-latex,thick] (\platinenX2,4*\i+0.9) -- (\motorreglerX1,4*\i+0.9) node[at start,left] {\scriptsize PWM-M\i} node[right] {\scriptsize PWM};
    \draw[thick] (\platinenX2,4*\i+0.6) -- (\motorreglerX1,4*\i+0.6) node[at start,left] {\scriptsize +5V} node[right] {\scriptsize VCC};
    \draw[thick] (\platinenX2,4*\i+0.3) -- (\motorreglerX1,4*\i+0.3) node[at start,left] {\scriptsize GND} node[right] {\scriptsize GND};

    \draw[-latex,thick] (\motorreglerX1,4*\i-0.4) -- (\platinenX2,4*\i-0.4) node[at end,left] {\scriptsize Serial-M\i} node[at start, right] {\scriptsize Serial};
    \draw[thick] (\platinenX2,4*\i-0.7) -- (\motorreglerX1,4*\i-0.7) node[at start,left] {\scriptsize GND} node[right] {\scriptsize GND};
    }

    
    % Servoansteuerung
    \node [thick, fit={(2.9,2.2) (\platinenX2,-1)},inner sep=0, align=center,shift={(0.4,0.8)}] {Platine Servoansteuerung\\ siehe \ref{sec:Servoplatine}};
    \draw [thick] (\platinenX2,-0.7)  -- (\platinenX2+0.5,-0.7) node [at start,left]{\scriptsize +5V};
    \draw [thick] (\platinenX2,-0.4)  -- (\platinenX2+0.8,-0.4) node [at start,left]{\scriptsize GND};
    \draw [thick] (\platinenX2,-0.1)  -- (\platinenX2+1.1,-0.1) node [at start,left]{\scriptsize SCL};
    \draw [thick] (\platinenX2,0.2)  -- (\platinenX2+1.4,0.2) node [at start,left]{\scriptsize SDA};
    
    %I2C Bus Leitungen
    \foreach \i in {0,0.3,0.6,0.9}{
        \draw [thick] (\platinenX2+0.5+\i,-0.7+\i) to [short,-*] ++ (0,-1.1) to [short,-*] ++ (0,-1.5) to [short,-] ++ (0,-1.5);
    }
    
    %ADS1115 Boards
    \newarray\names
    \readarray{names}{+5V&GND&SCL&SDA}
    \foreach \i in {0,1}{
        \foreach \r in {0,1,2,3}{
            \tikzmath{int \t; \t = \r+1; int \lm; \lm = \i*4+\r;}
            \draw[thick] (\platinenX2+0.5+0.3*\r,-2-3*\i+0.2+0.3*\r) -- (11,-2-3*\i+0.2+0.3*\r) node [at end, right]{\scriptsize \names(\t)};
            \draw[-latex,thick,double] (13.5,-2-3*\i+0.2+0.3*\r) -- (14.5,-2-3*\i+0.2+0.3*\r) node [at start, left] {\scriptsize A\r} node[right] {\scriptsize LM35-\lm};
        }
        \draw [very thick] (11,-2-3*\i) rectangle (13.5,0-3*\i);
        \node [thick, fit={(11,-2-3*\i)(13.5,0-3*\i)},inner sep=0, align=center,shift={(0.0,0.6)}] {ADS1115};
    }

    %Servo Driver Board
    \draw [very thick] (4,-3.5) rectangle (8.5,-1.5);
    \node [thick, fit={(4,-3.5) (8.5,-1.5)},inner sep=0, align=center,shift={(0.0,0.5)}]{Servo Driver Board};
    \foreach \r in {0,1,2,3}{
            \tikzmath{int \t; \t = \r+1;}
            \draw[thick] (8.5,-3.5+0.2+0.3*\r) -- (\platinenX2+0.5+0.3*\r,-3.5+0.2+0.3*\r) node [at start,left]{\scriptsize \names(\t)};
    }
    \foreach \i in {0,1,2}{
        \tikzmath{int \t; \t = \i*4; int \se; \se = \i+1;}
        \draw [-latex,thick] (5+0.3*\i,-3.5) -- (5+0.3*\i,-4.5) node [at start,rotate=90,right]{\scriptsize CH\t} node [at end,rotate=90,left]{\scriptsize PWM-S\se}; 
    }

    % Lenker Platine
    \node [thick, fit={(\platinenX1,12+2.2) (\platinenX2,12-1)},inner sep=0, align=center,shift={(-0.3,-0.3)}] {Platine Lenkung\\ siehe \ref{sec:Lenkerplatine}}; % Beschriftung

    % Nextion Display
    \draw [very thick] (11,11-0.2) rectangle (15,12.7-0.2);
    \node [thick, fit={(11,11-0.2) (15,12.7-0.4)},inner sep=0, align=center,shift={(0.4,-0.0)}]{Nextion Display};

    \newarray\Values
    \newarray\ValuesScreen
    \readarray{Values}{RX&TX&+5V-Disp.&GND}
    \readarray{ValuesScreen}{TX&RX&VCC&GND}
    \foreach \i in {0,1,2,3}{
        \tikzmath{int \t; \t = \i+1;}
        \draw [thick] (\platinenX2,12-0.8+\i*0.3) -- (11,12-0.8+\i*0.3) node [at start,left] {\scriptsize \Values(\t)} node [right] {\scriptsize \ValuesScreen(\t)};
    }

    %GPS Module
    \draw [very thick] (11,13-0.2) rectangle (15,14.4);
    \node [thick, fit={(11,13-0.2) (14.5,14)},inner sep=0, align=center,shift={(0.5,0.0)}]{Neo-8M GPS};

    \newarray\Values
    \newarray\ValuesScreen
    \readarray{Values}{D9&D8&+3.3V&GND}
    \readarray{ValuesScreen}{RX&TX&VCC&GND}
    \foreach \i in {0,1,2,3}{
        \tikzmath{int \t; \t = \i+1;}
        \draw [thick] (\platinenX2,12+2-0.9+\i*0.3) -- (11,12+2-0.9+\i*0.3) node [at start,left] {\scriptsize \Values(\t)} node [right] {\scriptsize \ValuesScreen(\t)};
    }

    % Analogue Inputs are drawn at the beginning so that double arrows look ok


\end{circuitikz}
}